% twocolumn を使うと2段組になる

%\documentclass[a4j,twocolumn]{jsarticle}        % -> platex
%\documentclass[a4j,twocolumn]{ujarticle}       % -> uplatex
\documentclass[uplatex]{jsarticle}   % -> uplatex + jsarticle

\usepackage{resume} % 他パッケージ,専用コマンド,余白の設定が書かれている

%%%%%%%%%%%%%%%%%%%%%%%%%%%%%%%%%%%%%%%%%%%%%%%%%%%%%%%%%%%%%%%%%%%%%%%%
% ヘッダ: イベント名,日付,所属,タイトル,氏名
%%%%%%%%%%%%%%%%%%%%%%%%%%%%%%%%%%%%%%%%%%%%%%%%%%%%%%%%%%%%%%%%%%%%%%%%

\pagestyle{plain}
\newcommand{\comment}[1]{}
\begin{document}
\twocolumn[
\beginheader{令和3年度 コンピュータサイエンス学部 最終発表}{2022}{2}{2}{井上 研究室}
\title{映像投影によるマスク下表情の可視化システム}
\author{C0118009 荒川 翔太郎 (Shotaro Arakawa) }
\endheader
]

\vspace{3mm}

 % 本番用ページ番号オフセット
\setcounter{page}{1}

%---------------------------------------------------------------------------
% 本文
%---------------------------------------------------------------------------


\section{はじめに}
人はぬいぐるみに対して声をかけたり抱きしめたりすることで安らぎを感じる.
この癒しの効果は,単身世帯が増え,高いストレスに晒される現代社会においてますます重要になっている.
ぬいぐるみは本来,綿や蕎麦殻などの詰め物を布で包んで縫い合わせ,特定のキャラクターの形に成型した物である.
しかし,その愛らしい見た目や柔らかな触感が先述の癒し効果を提供する一方で,人の問いかけや接触に対して動的な反応ができない.

この課題に対し,ぬいぐるみの内部に動力機構を埋め込むことで人とぬいぐるみとの間のインタラクションを高度化する試みが多数実践されている\cite{PINOKY}.
これらのアプローチにはぬいぐるみの解体が必要であり,癒し効果が低下してしまう可能性がある.

そのため,ぬいぐるみと同じ見た目をした3DCGを仮想空間に提示し,その仮想のぬいぐるみを変形することでインタラクションを高度化する試みが提案された\cite{仮想的な外見}.
しかしこのアプローチではぬいぐるみの本質的な特性の一つである柔軟性については扱っておらず,仮想空間の柔軟性を変更することによって癒し効果を増減できる可能性がある.

これらから,現実のぬいぐるみの内外に装置を必要とせず,仮想世界にあるぬいぐるみの柔軟性を視覚的に変更するシステムが必要であると考えた.
%により癒し効果を増減させることが期待される.


\section{仮想柔軟性を有するぬいぐるみシステム}
本システムの概要を\figref{fig:gaiyou}に示す.
ユーザはヘッドマウントディスプレイ(HMD)と両手の甲にVIVE Trackerを装着し,その正面にぬいぐるみを配置する.

ユーザがぬいぐるみを両手で押し込んだ際,その直角方向にひずみが生じることをポアソン効果といい,
その効果を仮想空間にあるぬいぐるみに適応させることによって柔軟性を提示する.

%ユーザが左右からぬいぐるみを押し込む動作をした際,仮想ぬいぐるみの伸縮率を変更することによって錯覚を起こす.
本システムでは3DCGを操作しているため,愛着のあるぬいぐるみをそのまま使用することができる.

\begin{figure}[t]
    \centering
    \fbox{\includegraphics[width=0.8\linewidth]{fig/システム概要(実写).png}}
    \caption{システム概要}\label{fig:gaiyou}
\end{figure}

\section{プロトタイプシステム}
\figref{fig:kousei}にシステム構成を示す.
本研究では提案したシステムのプロトタイプを開発した.
実装をハードウェアとソフトウェアに分けて説明する.
\begin{figure}[t]
    \centering
    \fbox{\includegraphics[width=0.9\linewidth]{fig/システム構成.pdf}}
    \caption{システム構成}\label{fig:kousei}
\end{figure}

    \subsection{ハードウェア}
    HMDとVIVE Trackerはベースステーションから位置情報を検出し,PCのUnityにその位置情報を送信する.
    それらの位置情報をPCに送ることにより,仮想空間におけるユーザの視点と手の3DCGをリアルタイムで反映させることが出来る.
    
    \figref{fig:douki}に同期させている様子を示す.
    このように現実の手の動きに合わせて手の3DCGを動かすことにより,没入感を向上させた.

    最後に,手の3DCGと仮想空間のぬいぐるみが反映された映像をPCからHMDへ送信することにより,ユーザにぬいぐるみインタラクションを提供する.

    \begin{figure}[t]
        \centering
        \fbox{\includegraphics[width=\linewidth]{fig/手位置同期モジュール.pdf}}
        \caption{現実空間と仮想空間における手の同期}\label{fig:douki}
    \end{figure}

    \subsection{ソフトウェア}
    \figref{fig:sinshuku}に柔軟性を変更した仮想ぬいぐるみを示す.
    本研究ではポアソン効果のうち,垂直方向のみを取り扱った.
    現実空間のぬいぐるみの伸びを1とし,その変化量を増減させることによって視覚的柔軟性を提示する.

    \begin{figure}[t]
        \centering
        \fbox{\includegraphics[width=\linewidth]{fig/仮想ぬいぐるみ伸縮モジュール.png}}
        \caption{柔軟性を変更した仮想ぬいぐるみ}\label{fig:sinshuku}
    \end{figure}


\section{評価実験}
開発したプロトタイプシステムを用いて評価実験を行った.
本実験では20代男性8名に協力してもらい,ユーザが仮想ぬいぐるみの見た目に対して許容可能な仮想柔軟性について調査を実施した.

仮想柔軟性の上限及び下限を調べるため,仮想ぬいぐるみの柔軟性を0.1ずつ増減し,ユーザが不自然さを感じたところで中止した.
また本実験においては垂直方向に対する仮想柔軟性の増減のみを行った.

実験終了後,「初期状態」「上限の一つ手前の状態」「下限の一つ手前の状態」のうち,どの状態が一番触れていたいと感じたかをアンケートした.


\section{結果および考察}
本システムで使用したぬいぐるみを所持している人が1名おり,他の被験者と明らかに数値が変わったため除外した.
そのため被験者7名における結果を示す.

    許容可能な仮想柔軟性の上限を\tabref{tab:result1}に,下限を\tabref{tab:result2}に示す.
    上限における平均値が1.88,標準偏差は0.16であった.
    下限においては平均値が0.61,標準偏差は0.16であった.
    このことから,本システムのプロトタイプを用いた場合,仮想柔軟性の許容される範囲は0.45倍$\sim$ 2.04倍であった.
    

    \begin{table}[t]
        \caption{\label{tab:result1}許容できる仮想柔軟性の上限}
        \centering
        \begin{tabular}{c|c|c|c|c|c|c}
            \hline
            \hline
            仮想柔軟性 &1.6 &1.7 &1.8 &1.9 &2.0 &2.1 \\
            \hline
            人数 &1 &0 &2 &1 &2 &1 \\
            \hline
        \end{tabular}
    \end{table}
    
    \begin{table}[t]
        \caption{\label{tab:result2}許容できる仮想柔軟性の下限}
        \centering
        \begin{tabular}{c|c|c|c|c|c}
            \hline
            \hline
            仮想柔軟性 &0.4 &0.5 &0.6 &0.7 &0.8 \\
            \hline
            人数 &2 &0 &2 &1 &2\\
            \hline
        \end{tabular}
    \end{table}


    実験終了後のアンケート結果を\tabref{tab:result3}に示す.
    半数以上が上限の一つ手前の状態を選択し,下限の一つ手前を選択する人はいなかった.
    そのため正の仮想柔軟性に対して寛容であることが示唆された.
    

    %一方で自由記述欄に「一番の違和感は場所が固定されていること」「潰れ方の種類が少ない」という意見があった.
    %さらに同じぬいぐるみを所持している理由から除外した被験者の自由記述欄には「家に同じぬいぐるみを持っているので頭の中のイメージとの相違が大きい」という意見があった.
    %この問題を解決するためには,より現実での触れ合いを再現したシステムを開発することが必要である.

    \begin{table}[t]
        \caption{\label{tab:result3}どの状態が一番触っていたいと感じたか}
        \centering
        \begin{tabular}{c|c}
            \hline
            \hline
            状態 &人数 \\
            \hline
            初期状態 &3 \\
            上限の一つ手前の状態 &4 \\
            下限の一つ手前の状態 &0 \\
            \hline
        \end{tabular}
    \end{table}

\section{おわりに}
本研究では,ポアソン効果を用いて仮想ぬいぐるみに柔軟性を付与するとで,ぬいぐるみインタラクションを向上させるシステムの提案と実装を行った.

その結果,本システムのプロトタイプにおいて仮想柔軟性の許容される範囲は 0.45倍$\sim$ 2.04 倍となり,正の仮想柔軟性に対しては寛容であることが示唆された.
%現在のプロトタイプでは垂直方向に対するポアソン効果しか再現されておらず,人とぬいぐるみとのインタラクションが 1 種類しかできていない.
%これにより仮想ぬいぐるみに伸縮率以外の不自然さを感じてしまい,評価実験に影響を及ぼした可能性がある.

今後の課題として,水平方向に対するポアソン効果も適応させること,手の認識率を高めること,固定された場所ではなく自由に持ち運びできることが挙げられる.

%---------------------------------------------------------------------------
% 本文終わり
%---------------------------------------------------------------------------

 % 参考文献
\bibliographystyle{junsrt}
\bibliography{ref}


\end{document}


%-----------------------------------------------------
% テンプレート
%------------------------------------------------------------------------------

%-----------
%% 箇条書き
%-----------
%\begin{itemize}
% \item
%\end{itemize}

%-------------------
%% 番号付き箇条書き
%-------------------
%\begin{enumerate}
% \item
%\end{enumerate}

%-----------
%% 図の表示
%-----------
%\begin{figure}[H]
% \centering
% \includegraphics[clip,width=7cm]{hoge.eps}
% \caption{図タイトル}\label{fig:hoge}
%\end{figure}

%-----------
%% 図の参照
%-----------
%\figref{fig:hoge}

%-----------
%% 表の作成
%-----------
%\begin{table}[H]
% \centering
% \caption{表タイトル}\label{tab:fuga}
% \begin{tabular}{|c|c|c|}\hline
%  hemo & piyo & fuga \\ \hline
%  hemo & piyo & fuga \\ \hline
% \end{tabular}
%\end{table}

%-----------
%% 表の参照
%-----------
%\tabref{tab:fuga}

%-----------
%% 参考文献
%-----------
%\begin{thebibliography}{9}
% \bibitem{piyo} 参考文献
%\end{thebibliography}

%-----------------
%% 参考文献の参照
%-----------------
%\cite{piyo}
